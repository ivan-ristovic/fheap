% !TEX encoding = UTF-8 Unicode

\documentclass[a4paper]{article}

\usepackage{color}
\usepackage{url}
\usepackage[utf8]{inputenc} % make weird characters work
\usepackage{graphicx}
\usepackage[english,serbian]{babel}
\usepackage[unicode]{hyperref}
\usepackage{amsthm}
\usepackage{amssymb}
\hypersetup{colorlinks,citecolor=green,filecolor=green,linkcolor=blue,urlcolor=blue}
\usepackage{listings}
\definecolor{codegreen}{rgb}{0,0.6,0}
\definecolor{codegray}{rgb}{0.5,0.5,0.5}
\definecolor{codeblue}{rgb}{0.0,0,0.82}
\lstdefinestyle{mystyle}{
    numbers=left,
    numberstyle=\scriptsize,
    numbersep=8pt,
    commentstyle=\color{codegray},
    keywordstyle=\color{codegreen},
    numberstyle=\tiny\color{codeblue},
    stringstyle=\color{codegreen},
    basicstyle=\ttfamily\footnotesize,
    breakatwhitespace=false,
    breaklines=true,
    captionpos=b,
    keepspaces=true,
    showspaces=false,
    showstringspaces=false,
    showtabs=false,
    tabsize=4,
    xleftmargin=3em,
    framexleftmargin=1.5em
}
\lstset{style=mystyle}


\theoremstyle{plain}
\newtheorem{thm}{Teorema}[section] % reset theorem numbering for each chapter
\theoremstyle{definition}
\newtheorem{defn}[thm]{Definicija} % definition numbers are dependent on theorem numbers
\newtheorem{exmp}[thm]{Primer} % same for example numbers


\begin{document}

\title{Fibona\v{c}ijev hip\\ \small{Seminarski rad u okviru kursa\\Konstrukcija i analiza algoritama 2\\ Matematički fakultet}}

\author{\href{mailto:ivan_ristovic@math.rs}{Ivan Ristovi\'c}\\\href{mailto:mi14042@matf.bg.ac.rs}{Milana Kovacevi\'c{}}}
\date{januar 2019.}

\maketitle

\abstract{
    Fibona\v{c}ijev hip je struktura podataka osmi\v{s}ljena sa ciljem da pobolj\v{s}a vreme potrebno za operacije nad hipovima. Pru\v{z}aju bolje amortizovano vreme izvr\v{s}avanja nego ve\'c{}ina drugih prioritetnih redova, uklju\v{c}uju\'c{}i binarni i binomni hip. Fibona\v{c}ijev hip je osmi\v{s}ljen od strane Fredman-a i Tarjan-a 1984. godine i publikovan 1987. Ime je dobio po Fibona\v{c}ijevim brojevima, koji se koriste u analizi slo\v{z}enosti operacija. Koriste\'c{}i Fibona\v{c}ijev hip, mogu\'c{}e je unaprediti vremena izvr\v{s}avanja velikog broja poznatih algoritama kao \v{s}to je Dijsktrin algoritam. Pru\v{z}amo implementaciju Fibona\v{c}ijevog hipa u programskom jeziku \emph{Python}, sa interfejsom jednostavnim za upotrebu i testiranje. Takodje u ovom radu testiramo vreme izvr\v{s}avanja operacije \emph{decrease-key} kako bismo eksperimentalno pokazali amortizovano vreme izvr\v{s}avanja ove operacije od $O(\log{n})$.
}

\tableofcontents

\newpage

\section{Uvod}
\label{sec:Uvod}

test \cite{Book} \cite{Slides}




\section{Zaključak}
\label{sec:Zakljucak}


\addcontentsline{toc}{section}{Literatura}
\appendix
\bibliography{references}
\bibliographystyle{plain}

%\appendix
%\section{Dodatak}


\end{document}
